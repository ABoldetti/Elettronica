\pagebreak
\subsection{SiPm}
La parte finale della caratterizzazione della scheda richiede lo studio del rilevatore al suo interno, ovvero un SiPm (Silicon Photo Multiplier).

Questo rilevatore presenta al suo interno il sistema composto da SPAD e resistenze di quenching che permette la creazione di valanghe di elettroni controllate all'interno del rilevatore fino ad ottenere una risoluzione al singolo fotone.

\subsubsection{Setup}

Siccome la tensione generata dal SiPm è troppo bassa per la risoluzione dell'ADC è necessario applicare gli stadi di amplificazione studiati nei punti seguenti come mostrato in figura:

\begin{figure}[!h]
    \centering
    \includegraphics[width=0.5\linewidth]{assets/SiPm/SiPm_Stadi_Amp.png}
    \caption{Stadi di amplificazione}
    \label{fig:SiPm stadi di amp}
\end{figure}

Siccome i tempi di esistenza del segnale sono molto bassi, nell'ordine dei 100-200ns, risulta difficile ottenere una misurazione accurata con l'ADC che è limitato a campionamenti nell'ordine dei 200ns. Per questo è stato usato un sistema di Peak Hold:

\begin{figure}[!h]
    \centering
    \includegraphics[width=0.5\linewidth]{assets/SiPm/SiPm_Peak_hold.png}
    \caption{Sistema di Peak Hold}
\end{figure}

All'uscita del peak hold la tensione viene mantenuta per un tempo sufficiente per l'acquisizione dell'ADC.

Come mostrato in figura \ref{fig:SiPm stadi di amp}, il rilevatore che si comporta come diodo è inserito in polarizzazione inversa nella scheda in maniera tale da sfruttarne la tensione di breakdown. Per comprendere la polarizzazione, è stata impostata inizialmente sull'alimentatore una tensione bassa e corrente limitata per controllare la polarizzazione corretta. Avendo inserito nei suoi pin la fonte luminosa a LED, è stata sigillata la zona contenente fonte luminosa e rilevatore. Successivamente, la tensione fornita al SiPm è stata portata a +3V di Overvoltage per permettere l'avvento della valanga di elettroni nello SPAD.

Fornendo al LED una tensione impulsata di 100 ns a circa 3.4 V è stato possibile osservare la rilevazione dei fotoni:

\begin{figure}[!h]
    \centering
    \includegraphics[width=0.5\linewidth]{assets/SiPm/SiPm.png}
    \caption{Prima visualizzazione dei segnali con persistance di 2 sec}
\end{figure}

E evidente la presenza di diverse zone con elevata densità di rilevazioni
che corrispondono a 1 fotone, 2 fotoni, 3 fotoni etc...

\subsubsection{Dark Count Rate}
Siccome gli elettroni nello spad sono soggetti ad agitazione termica, è possibile che un elettrone riesca casualmente a liberarsi anche senza la presenza di un fotone, portando a una valanga non desiderata. Questo è noto come DCR (Dark Count Rate).

Per caratterizzare il SiPm è utile avere una stima di questo fenomeno. Per questo è stato impostato l'oscilloscopio in modalità di conteggio dei segnali in arrivo. Impostata una soglia di trigger di circa mezzo fotone è stata fatta partire la misura di tempo per 40000 segnali a tensioni di overvoltage differenti.

\begin{figure}[!h]
    \centering
    \includegraphics[width=0.6\linewidth]{assets/SiPm/SiPm_DCR.pdf}
    \caption{Tempi per conteggio di 40000 eventi, (\href{https://github.com/Yedi278/Esperimentazioni-Elettronica/tree/main/SiPm/Caratterizzazione\%20Hamamatsu}{link dati})}
\end{figure}

Nella zona prossima alla tensione di \textit{break down} il rate risulta molto basso, si suppone dovuto alla mancanza di campo elettrico fornito agli elettroni che non sono in grado in certi casi di produrre la valanga. Per tenisoni prossime a quelle di \textit{overvoltage} consigliate, ovvero $V_{OV} = 54.9V$, il rate si comporta in maniera approssimativamente lineare. Siccome il rate cambia di circa 31 ms/V possiamo approssimare questo rate ad un valore costante di circa \textbf{DCR = 158±38KHz}.
Questo valore risulta ragionevole con i valori inclusi nel manuale.