\pagebreak

\subsection{Programmazione}
Per programmare sul microcontrollore è stato usato il programma STM32CubeIDE, un IDE fornito dalla STMicroelectronics. Il linguaggio di programmazione usato è C99. All'interno dell'IDE sono inserite delle librerie che permettono di interfacciarsi con le periferiche del microcontrollore, le quali contengono gli indirizzi dei registri e le maschere per collegare ogni bit dei registri al funzionamento sulla periferica.\\

I registri prendono il nome dalla periferica che controllano e ogni registro è diviso in sottoregistri a cui si può accedere usando i puntatori.

Per modificare i registri è consigliato usare maaschere, quindi dei numeri binari che, se applicate delle operazioni di AND e OR permettono di andare a modificare solo determinati bit.

\subsubsection{Struttura del programma}
Il programma si divide in diversi file:
\begin{itemize}
    \item \textbf{main.c}: Il file principale del programma, contiene la funzione main() e le funzioni di inizializzazione delle periferiche. Il compilatore lancia il programma a partire da questo file.
    \item \textbf{stm32h7xx_it.c}: Il file contenente tutti gli interrupt abilitati. Ogni interrupt è una funziione diversa che non comunica direttamente con il resto.
    \item \textbf{Librerie create dall'utente}: File contenenti le funzioni create dall'utente per interfacciarsi con le periferiche.
\end{itemize}

Questi sono solo alcuni dei file presenti nel progetto, ma sono quelli che effettivamente modificati dall'utente.



