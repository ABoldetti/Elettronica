\pagebreak
\subsection{OP27}

\begin{wrapfigure}{R}{.4\linewidth}
    \centering
    \includegraphics[width=\linewidth]{assets/OP27/OP27_datasheet.png}
    \caption{Feature OP27}
\end{wrapfigure}

Il primo amplificatore che vogliamo studiare è l' OP27 che dai datasheet risulta essere un amplificatore a singolo polo dominante ovvero della forma:
\[A(s) = \frac{A_o}{1+s\tau_A}\]
Le cui caratteristiche sono descritte in dettaglio nei datasheet.



In particolare è segnata una \textbf{Bandwidth} di circa \textbf{8Mhz} che risulterà essere il dato di nostro interesse.

L'amplificatore può essere configurato in due modalità, ovvero: configurazione invertente e configuraizone non invertente come mostrato in figura:

\begin{figure}[!h]
    \centering
    \begin{subfigure}[b]{0.45\textwidth}
        \includegraphics[width=0.9\textwidth]{assets/scheda analogica/invertente.png}
        \caption{configurazione invertente}
    \end{subfigure}%
    \begin{subfigure}[b]{0.45\textwidth}
        \includegraphics[width=0.9\textwidth]{assets/scheda analogica/non_invertente.png}
        \caption{configurazione non invertente}
    \end{subfigure}
\end{figure}

Queste risultano entrambe utili a seconda dello stadio di amplificazione. La configurazione invertente permette un guadagno maggiore a costo di invertire il segnale.
La configurazione non invertente invece non altera il segnale ma applica semplicemente un'amplificazione.

\subsubsection{Configurazione Non Invertente}

In questa configurazione abbiamo un guadagno di Gain $\frac{1}{\beta} = \frac{R_1+R_2}{R_1}$ nel caso di amplificatiore \textbf{ideale}.
Collegando nella scheda resistenze $R_s = 0.5 \Omega$, $R_1 = 2.2k\Omega$ e $R_2 = 2.2k\Omega$ ci aspettiamo un gain di 2.

Siccome l'amplificatore reale dista dal modello ideale di un polo complesso, è necessario considerare il fattore $\frac{A(\omega)}{\beta}$ al posto di $\frac{1}{\beta}$ nella sua funzione di trasferimento dove A($\omega$) denota la dipendenza dalla frequenza. Per poter caratterizzare questo comportamento in frequenza vogliamo trovare la \textbf{banda} dell'amplificatore.

\'E possibile utilizzare tre modalità diverse:
\begin{enumerate}
    \item Analisi di onde sinusoidali
    \item Analisi su un'onda quadra
    \item Analisi dello spettro
\end{enumerate}

\paragraph{Analisi sinusoidale con $R_1=2.2k\Omega$ e $R_2=2.2k\Omega$:}

Per questa modalità è stato impostato un segnale sinusoidale all'ingresso non invertente dell'OP27 ad una ampiezza nota e osservato tramite oscilloscopio la risposta del circuito nel dominio del tempo alle varie frequenze. Per il calcolo della banda è sufficiente trovare la frequenza per cui il gain diminuisce di 3dB, ovvero perde il 30\% della sua ampiezza.

\begin{figure}[!h]
    \centering
    \includegraphics[width=0.5\linewidth]{assets/OP27/Non Invertente/Sin_2k2.pdf}
    \caption{Segnali di uscita dall'amplificatore per frequenza a risposta piatta e a -3dB, (\href{https://github.com/Yedi278/Esperimentazioni-Elettronica/tree/main/-\%20OPAMP/OP27/Non-Invertente/R1\%3D2.2k\%2CR2\%3D2.2k}{link dati})}
\end{figure}

Come frequenza bassa è stato usato un segnale sinusoidale di circa 1KHz mentre la frequenza per cui il segnale perde 3dB è di circa 3.9Mhz.
Per la banda si trova: \bm{$Banda = \frac{\omega_F}{\beta} = 7.8 Mhz$}.

\paragraph{Analisi in onda quadra a $R_1=2.2k\Omega$ e $R_2=2.2k\Omega$:}

Per questa modalità è stata impostata nel generatore un'onda quadra e tramite la modalità di misura dell'oscilloscopio sempre nel dominio del tempo, è stato acquisito il valore di \textbf{rise time} della risposta dell'amplificatore. Con rise time si intende il tempo che impiega il segnale a passare dal 10\% al 90\% della sua ampiezza.

\begin{figure}[!h]
    \centering
    \includegraphics[width=0.5\linewidth]{assets/OP27/Non Invertente/Square_2k2.pdf}
    \caption{Risposta ad onda quadra, (\href{https://github.com/Yedi278/Esperimentazioni-Elettronica/tree/main/-\%20OPAMP/OP27/Non-Invertente/R1\%3D2.2k\%2CR2\%3D2.2k}{link dati})}
\end{figure}
\pagebreak

Il rise time misurato dall'oscilloscopio è di 79.5 ns che corrispondono ad una frequenza a -3dB di:
$$f_{-3dB} = \frac{0.35}{t_r} = 4.4Mhz$$ e dunque una \textbf{banda di 8.8Mhz}.

% \paragraph{Nota}
% \'E possibile notare nella risposta in onda quadra un comportamento insolito, infatti è presente un notevole overshoot (circa 20\%) nella risposta dell'amplificatore.


% Siccome un amplificatore a singolo polo non dovrebbe mostrare questo fenomeno è necessario supporre che la presenza di singolo polo dominante sia errata. Una opzione è quella di considerare anche la presenza di capacità parassite in parallelo alle resistenze del circuito. 

% Tramite le formule presenti nel pdf al \href{https://pessina.mib.infn.it/Corsi_del_III_anno/CorsoStrumentazioneElettronica/Corso_2425/Amplificatori_reazionati_analisi_tempo_frequenza_B_2425.pdf}{link} e discussa più in dettaglio nel file \href{https://github.com/Yedi278/Esperimentazioni-Elettronica/blob/main/Analisi_Overshoot.mlx}{matlab}, si ottiene: 

% \begin{equation}
%     T(s) = \frac{R_1+R_2}{R_1} \frac{1+s\tau_2}{(1+s\tau_1)(1+s\tau_A)} A_o
% \end{equation}

% Dunque si può osservare un andamento a doppio polo che potenzialmente potrebbe portare a spiegare l'andamento ma come mostrato più in dettaglio nel file matlab il valore delle capacità parassite dovrebbe superare notevolmente l'odine dei pico-Farad, che risulta insolito. \'E dunque necessario escludere anche questa possibilità.

% Non resta altro che supporre non trascurabile l'influenza di altri poli all'interno dell'OP27 per frequenze maggiori del polo dominante.


%\textcolor{red}{Bolde}
\begin{flushleft}

\colorbox{notebox}{
\begin{minipage}[]{\textwidth}

    Notiamo come in risposta all'onda quadra, l'amplificatore presenti un'overshoot non trascurabile: circa il 20\% rispetto al valore aspettato.
Questo può essere dovuto a diversi fattori:
\begin{itemize}
    \item la presenza di capacità parassite all'interno dell'amplificatore
    \item la presenza di un secondo polo nell'amplificatore che abbia un influenza inaspettata anche a frequenze basse
\end{itemize}
Prendendo in considerazione l'ipotesi di capacità parassite all'interno dell'amplificatore, si può modificare la formula della funzione di trasferimento come segue
\begin{equation}
    T(s) = \frac{R_1+R_2}{R_1} \frac{1+s\tau_2}{(1+s\tau_1)(1+s\tau_A)} A_o
\end{equation}
Il quale motiva la presenza del doppio polo. Sfortunatamente, questa ipotesi va scartata in quanto dai dati risulta che le capacità dovrebbero essere molto maggiori del pico Farad, il quale risulta insolito.

Risulta quindi la seconda ipotesi, ovvero che l'OP27 non sia accettabile come amplificatore a polo singolo ma è necessario considerare anche i poli di ordine superiore per avere delle misure coerenti con la teoria
\end{minipage}
}
\end{flushleft}



%\textcolor{red}{fine Bolde}


\begin{figure}[!h]
    \centering
    \includegraphics[width=0.3\linewidth]{assets/OP27/OP27_Phase_Margin.png}
    \caption{Matgine di fase OP27}
\end{figure}

\begin{flushleft}
    

\colorbox{notebox}{
\begin{minipage}[]{\textwidth}
Guardando nei datasheet la fase dell'amplificatore si nota che a gain bassi ci si può ridurre nella condizione con margine di fase ridotto. In questa zona possiamo dedurre che si abbiano effetti non previsti come l'andamento instabile osservato nelle misure sperimentali.

Inoltre questa ipotesi a differenza delle altre è consistente con l'andamento osservato per l'overshoot in figura \ref{fig:OP27 overshoot non invertente} che diminuisce all'aumentare del gain assegnato.
\end{minipage}
}
\end{flushleft}
\paragraph{Nota 2}
Siccome in alcuni casi il generatore di onda quadra non risulta perfetto è necessario considerare il rise time di quest'ultimo:
\[t_r = \sqrt{t_m^2 + t_g^2}\]
dove $t_m$ è il rise time misurato e $t_g$ il rise time del generatore da solo.

\pagebreak

\begin{wrapfigure}{R}{.4\linewidth}
    \centering
    \includegraphics[width=\linewidth]{assets/OP27/Non Invertente/FFT_2k2.pdf}
    \caption{ (\href{https://github.com/Yedi278/Esperimentazioni-Elettronica/tree/main/-\%20OPAMP/OP27/Non-Invertente/R1\%3D2.2k\%2CR2\%3D2.2k}{link dati})}    
\end{wrapfigure}

\paragraph{Analisi Spettrale per  $R_1=2.2k\Omega$ e $R_2=2.2k\Omega$:}
L'ultima misura sfrutta il rumore bianco, ovvero rumore che ha potenza spettrale piatta e dunque uguale a ogni frequenza.
Avendo selezionato nel generatore un rumore bianco, è stata utilizzata la funzione FFT dell'oscilloscopio per ottenere lo spettro dell'amplificatore.





\'E possibile in questo caso utilizzare manualmente due cursori per individuare il gain a risposta piatta e dato questo, individuare la frequenza a -3dB da quest'ultima.


 Nella misura otteniamo una risposta piatta a circa -45dB e una $f_{-3dB} = 3.8Mhz$ da cui una \textbf{banda di 7.6Mhz}.

\paragraph{Cambio Resistenze}
Variando la resistenza $R_2$ è possibile cambiare il gain del circuito e studiare l'andamento della banda.
Ripetendo il processo eseguito prima per valori di $R_2 = \{2.2k\Omega, 8.2k\Omega, 10k\Omega, 17.2k\Omega\}$ è possibile notare il seguente andamento: 

\begin{figure}[!h]
    \centering
    \includegraphics[width=.5\linewidth]{assets/OP27/Non Invertente/Non_Invert_Dati_tot.pdf}
        \captionof{figure}{Rappresentazione grafica dell'andamento per configurazione Non-invertente}
        \label{fig:OP27 overshoot non invertente}
\end{figure}

\begin{table}[!h]
    \centering
    \begin{tabular}{|l|l|l|l|l|}
            \hline
            \textbf{Gain {[}1{]}} & \textbf{Sin {[}Mhz{]}} & \textbf{Square {[}MHz{]}} & \textbf{Fft {[}MHz{]}} & \textbf{Overshoot {[}MHz{]}} \\ \hline
            2             & 7.76         & 8.8             & 7.42         & 19.8               \\ \hline
            4.73          & 10.4         & 9.3             & 10.4         & 4.45               \\ \hline
            5.55          & 10.92        & 10.52           & 11           & 6.62               \\ \hline
            9.91          & 9.3          & 10.2            & 9.4          & 0.83               \\ \hline
        \end{tabular}
        \captionof{table}{Rappresentazione a tabella dei dati per configurazione Non-invertente}
\end{table}

Il valor medio di banda ricavato è di \textbf{9.6MHz}.

\paragraph{Conclusioni}
 \'E stato possibile con tre metodi diversi eseguire un'analisi della risposta del circuito alle frequenze in ingresso. In particolare è stato riscontrato un polo complesso che porta ad una larghezza di \textbf{banda $\sim 9.6$Mhz} che risulta in accordo entro un 20\% con il valore dei datasheet.

\subsubsection{Configurazione Invertente}

L'altra configurazione interessante per un amplificatore è quella invertente in cui il segnale entra nell'ingresso negativo $V^-$.
 In questa configurazione si ha un guadagno di $$Gain = -\frac{R_2}{R_1}$$

 Il procedimento per caratterizzare la banda risulta identico ed è possibile nelle tre modalità citate prima.

 Questo è un esempio di risposta osservato per resistenze di $R_1 = R_2 = 2.2k\Omega$

\begin{minipage}{\textwidth}

    \centering
    \begin{minipage}[b]{0.3\textwidth}
        \includegraphics[width=\linewidth]{assets/OP27/Invertente/Sin_17k8.pdf}
        \captionof{figure}{Analisi con segnale sinusiodale}
    \end{minipage}
    \hfill
    \begin{minipage}[b]{0.3\textwidth}
        \includegraphics[width=\linewidth]{assets/OP27/Invertente/Square_17k8.pdf}
        \captionof{figure}{Analisi con onda quadra (overshoot $<$ 20\%)}
    \end{minipage}
    \hfill
    \begin{minipage}[b]{0.3\textwidth}
        \includegraphics[width=\linewidth]{assets/OP27/Invertente/FFT_17k8.pdf}
        \captionof{figure}{Analisi nel dominio delle frequenze}
    \end{minipage}
    \hfill    
\end{minipage}

Da notare che l'uscita è invertita rispetto al segnale. Il resti dei dati è al link: (\href{https://github.com/Yedi278/Esperimentazioni-Elettronica/tree/main/-\%20OPAMP/OP27/Invertente}{link dati})

\paragraph{Risultato:}
Applicando il processo di misura per varie resistenze $R_2 = \{ 2.2k\Omega, 8.2k\Omega, 10k\Omega, 17.8k\Omega \}$ è possibile vedere l'andamento seguente:

\begin{figure}[!h]
    \centering
    \includegraphics[width=.5\linewidth]{assets/OP27/Invertente/OP27_Invert_Dati_tot.pdf}
    \captionof{figure}{Rappresentazione grafica dell'andamento per configurazione Non-invertente}
\end{figure}

\begin{table}[!h]
    \centering
    \begin{tabular}{|c|c|c|c|c|}
    \hline
    \textbf{Gain {[}1{]}} & \textbf{Sin {[}Mhz{]}} & \textbf{Square {[}MHz{]}} & \textbf{Fft {[}MHz{]}} & \textbf{Overshoot {[}MHz{]}} \\ \hline
    2.27                  & 8.8                    & 10.51                     & 9.12                   & 23.6                         \\ \hline
    3.73                  & 8.57                   & 8.35                      & 8.57                   & 11.7                         \\ \hline
    4.55                  & 8.69                   & 8.74                      & 9.28                   & 7.8                          \\ \hline
    8.05                  & 9.26                   & 9.74                      & 10.14                  & 2.05                         \\ \hline
    \end{tabular}
    \captionof{table}{Rappresentazione a tabella dei dati per configurazione Non-invertente}
\end{table}


Come per il caso non invertente si osserva una diminuizione dell'overshoot all'aumentare del gain nel circuito. 

\subsubsection{Conclusione}

L'unione dei dati presi tramite le due configurazioni porta al seguente andamento:

\begin{figure}[!h]
    \centering
    \includegraphics[width=\linewidth]{assets/OP27/Inv+Non-Inv.pdf}
    \caption{Valori di banda calcolati tramite le tre configurazioni per Invertente e Non-inv, (\href{https://github.com/Yedi278/Esperimentazioni-Elettronica/tree/main/-\%20OPAMP/OP27}{link dati})
    }
\end{figure}

\'E possibile notare che tra i due modelli si ottiene lo stesso andamento per l'overshoot che conferma l'ipotesi di dipendenza dai componenti interni all'amplificatore escludendo elementi esterni come capacità parassite.
\'E possibile inoltre notare una dispersione elevata delle misure. Questo si suppone dovuto ad errori di imprecisione nella presa dati (es cursori non perfettamente allineati o resistenze con tolleranze elevate).

Tramite i dati raccolti è possibile ottenere una stima della banda di \textbf{9.4 ± 0.9 MHz}. Il valore di riferimento di 8Mhz entra in 1.5 $\sigma$ dalla stima.

